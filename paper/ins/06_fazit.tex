\clearpage
\section{Fazit}
\label{sec:fazit}
Die Kundennähe konnte als ein entscheidender Wettbewerbsfaktor gezeigt werden. Erreichen lässt sie sich durch die Analyse der Kundendaten über Methoden der Customer Analytics in Verbindung mit dem CRM. Dabei zeigen sich für Unternehmen je nach Geschäftsmodell unterschiedliche Schwerpunkte und Herausforderungen auf.
Ferner stellt die Auswahl der Features aus Daten eine Herausforderungen dar, die je nach Unternehmensziel stark variieren kann, bietet dabei aber gleichzeitig Möglichkeiten für ein Business Intelligence System.

Die analytische Annäherung über Methoden wie dem Data Mining wird insbesondere bei großen Datenmengen bedeutsam.
Durch eine Segmentierung, ähnlich der des gezeigten Anwendungsbeispiels, kann durch die Berücksichtigung von Kundeninteressen und -eigenschaften eine effizientere Interaktion ermöglicht werden.
Dies kann Kosten des Kundenkontaktes verringern und gleichzeitig durch die verbesserte Ansprache zu einem größeren Unternehmenserfolg führen.

Die Customer Analytics ist für Unternehmen von hoher Relevanz.
Sie bedingt Umstrukturierungen und Veränderungen der Unternehmensstrategien, die langfristig den Bestand und Erfolg des Unternehmens sichern.
Damit ist die Customer Analytics ein unverzichtbarer Teil der Business Intelligence.