\section{Begriffsbestimmung und Einordnung}
\label{sec:begriffsbestimmung_einordnung}

\subsection{Bezug zu der Business Intelligence}
Die Customer Analytics ist ein Teilgebiet der Business Intelligence, die wiederum als die „Extraktion und Auswertung der unternehmensweit vorhandenen Daten und deren Umwandlung in für die Entscheider relevante Informationen“ \cite[S. 6]{hannig2002} durch verschiedene Softwarewerkzeuge bezeichnet werden kann.
Der Einsatz von Werkzeugen und Anwendungen der Business Intelligence soll zu einer „besseren Einsicht in das eigene Geschäft und damit zu einem besseren Verständnis in die Mechanismen der relevanten Wirkungsketten führen“ \cite[S. 32]{hannig2002}.

Dabei kommt dem internen Wissensmanagement als Voraussetzung für erfolgreichen Einsatz der Softwarewerkzeuge große Bedeutung zu. Dieses gliedert sich in die Teilbereiche der „Gewinnung von Wissen aus allen verfügbaren Quellen, die Strukturierung, Aufbereitung und Speicherung des generierten Wissens und die bedarfsgerechte Zurverfügungstellung des Wissens“ \cite[S. 16]{hannig2002}.
Je nach Perspektive umfasst der Begriff Business Intelligence unterschiedliche große Bereiche. Eine sehr weite Auslegung würde alle Tools umfassen, mittels denen operatives Datenmaterial gewonnen und ausgewertet werden kann.
Damit würden auch die Datenquellen selbst zu der Business Intelligence gezählt werden. Eine etwas engere Definition über ein analyseorientiertes Verständnis schließt diese aus, umfasst jedoch noch weitere Maßnahmen über Data Mining, Kennzahlen oder analytisches Customer Relationship Management (CRM), die in der engsten definitorischen Sichtweise nicht mehr beinhaltet sein würden.\cite{hannig2002}

\subsection{Konzeptualisierung und Definition}

Der Begriff Customer Analytics findet keine einheitliche Definition in der Literatur. Vielmehr wird sich diesem auf Basis verschiedener Kennzahlen angenähert, die sich wiederum auf Teilgebiete der Customer Analytics beziehen.

In diesem Abschnitt wird daher zunächst eine konzeptionelle Grundlage durch die Theorie informationeller Mehrwerte gegeben, im Anschluss daran eine Arbeitsdefinition aufgestellt und nachfolgend einige Kennzahlen vorgestellt, die eine Abgrenzung und ein besseres Verständnis des Themenbereiches ermöglichen.

\subsubsection{Die Theorie informationeller Mehrwerte}
Als Grundlage der Customer Analytics wird auf die Theorie informationeller Mehrwerte \cite{kuhlen1995} verwiesen. Kuhlen versteht als informationellen Mehrwert einen über den Grundwert hinaus gehenden Zusatzwert, der aber auch erst kundenseitig als solcher empfunden werden muss, um als ein solcher Mehrwert bezeichnet werden zu können.

Um eine Mehrwertleistung bieten zu können, braucht es demnach Wissensbestände, aus denen Informationen über Wünsche und Ansprüche der Nutzer bzw. Kunden abgeleitet werden können.

\subsubsection{Definition}
Der dieser Arbeit zugrunde liegende Begriff der Customer Analytics wird damit ausgehend von der Theorie informationeller Mehrwerte als ein Prozess verstanden, bei dem aus der Analyse der dem Unternehmen vorliegenden Kundendaten (neue) Erkenntnisse gewonnen werden, um das operationale und strategische Management daraufhin auszurichten. \cite{kumar2018}

Customer Analytics dient also der Erforschung der Kundengruppen und deren Bedürfnissen bzw. deren Zufriedenheit. Dafür werden Kundengruppen auf Basis ihres Verhaltens segmentiert, um zukünftiges Verhalten vorherzusagen oder Trends zu erkennen oder um Marketing- bzw. Vertriebsaktivitäten gezielt auf Kundengruppen ausrichten zu können. \cite{gartner2020}

Je nach Verständnis der Business Intelligence können Maßnahmen der Customer Analytics nur in den analyse-orientierten Teil der Business Intelligence eingeordnet werden und gehören damit nicht zu den Kernelementen der Business Intelligence.

\subsection{Praxisorientierte Anwendung der Customer Analytics}
Für die Erzeugung eines informationellen Mehrwertes im wirtschaftlichen Umfeld müssen auch in der Customer Analytics zunächst Kundeninteressen bekannt sein. Die dieser Ermittlung zugrunde liegenden Daten sind aus einer Vielzahl möglicher Interaktionen mit dem Kunden entstanden. Kunden können dabei Individuen, aber auch Gruppen oder Organisationen sein. Informationen über diese werden in einem Datenbanksystem abgelegt, welches gleichzeitig das CRM ermöglicht, auf das in nachfolgenden Abschnitten vertieft eingegangen wird.

Die anwendungsorientierte Customer Analytics beruht also zum einen auf den Datenbeständen über die Kunden aus den internen Systemen, welche sich in einem stetigen Veränderungsprozess befinden. Daher müssen diese Daten laufend angepasst bzw. aktualisiert werden, um verlässliche Aussagen treffen und daraufhin die Unternehmensstrategie auslegen zu können. Im digitalen Kontext können zum anderen als weitere Datenquelle Transaktionsdaten bzw. Daten aus dem Tracking des Navigationsverhaltens der Nutzer/Kunden verwendet werden, die durch Data Mining Techniken ausgewertet werden. Die Auswertung der anfallenden Datenmengen (Big Data) werden also im Zuge der der Customer Analytics ausgewertet. Daher wird in diesem Zusammenhang zunehmend der Begriff „Smart Data“ verwendet.

Aus dem Vorliegen von Big Data wird zugleich eine der beiden Herausforderungen der Customer Analytics \cite{choudhury2010} deutlich. Dem Unternehmen liegen nahezu unbegrenzte Möglichkeiten von Metriken vor, um eine Analyse durchzuführen. Choudhury et al. \cite{choudhury2010} bezeichnen dies als das „relevance“ Phänomen. Als eine weitere Schwierigkeit der Customer Analytics bezeichnen sie die „inference“. Diese resultiert aus dem Problem, relevante Komponenten für die Analyse nicht direkt aus dem existierenden Datenbestand entnehmen zu können, sondern aus weiteren Maßnahmen zu erschließen.

\subsection{Verwendung von Kennzahlen in der Customer Analytics}
Für eine Einschätzung und Vergleichsmöglichkeit individueller Kunden oder Kundengruppen werden in der Praxis häufig Kennzahlen der Customer Analytics eingeführt.

Diese Key Perfomance Indicators (KPIs) umfassen Dimensionen, wie den Umsatz, die Zeit zwischen Transaktionen, den Anteil der gekauften Produktkategorien, die Preissensitivität uvm. \cite{grigsby2016}. Auch hier werden die Ergebnisse aus dem Kundenverhalten abgeleitet.

Neben der Segmentierung von Kundengruppen bieten die Customer Lifetime Value (CLV), die Customer Profitability (CP) oder die Recency-Frequency-Monetary Methode (RFM-Methode) weitere Bewertungsmöglichkeiten. Die ersten beiden KPIs sind zur Messung von Unterschieden der Konsumenten geeignet \cite{pfeifer2005} und ergeben ein wertbasiertes Urteil über den jeweiligen Kunden.
Der Unterschied besteht darin, dass die CLV ein Urteil über Deckungsbeitrag eines Kunden zu einem gegebenen Zeitpunkt macht, wobei in der CP das Kosten-Nutzen-Verhältnis eines Kunden betrachtet wird.

Die RFM-Methode \cite{fader2004} wiederum kombiniert drei Faktoren: Recency \textit{(wann hat der Kunde zuletzt eingekauft?)}, Frequency \textit{(wie oft hat ein Kauf stattgefunden?)} und Monetary Value \textit{(wie viel Geld hat der Kunde ausgegeben?)}.

Diese wertbasierten Kundenklassifikationen lassen sich kombinieren \cite{fader2004}.
Aufgrund ihrer geringen Komplexität werden diese KPIs, insbesondere die RFM Methode, gerne in der alltäglichen Marketing-Praxis verwendet. Durch komplexere Techniken der Customer Analytics lassen sich jedoch weitere Unternehmensziele operationalisieren.