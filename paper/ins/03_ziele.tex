\section{Ziele der Customer Analytics}
\label{sec:ziele}

Durch ein tieferes Kundenverständnis können verschiedene direkt auf den Kunden ausgerichtete Disziplinen verbessert werden, dazu zählen Produkte und Services des Unternehmens, sowie das Marketing und ein verbessertes CRM. Somit lassen sich durch Customer Analytics ebenso Teilgebiete der „value strategies“ behandeln.

\subsection{Produkt-/Service-orientierte Optimierung}
Aus Erkenntnissen über Kundenbedürfnisse oder die Zufriedenheit der Kunden können Veränderungen von Produkten und Services bis hin zu vollständigen Prozessen des Unternehmens vorgenommen werden. Für eine Verbesserung der operationalen Exzellenz bzw. der Produktmarktführerschaft im Sinne der „value strategies“ muss ggf. ebenso die Unternehmensstrategie angepasst werden. Alle Prozesse sollten von Beginn an auf den Kunden und dessen möglicherweise veränderte oder individuelle Interessen angepasst sein.
Dies führt ebenfalls zu der Notwendigkeit der Orientierung aller Mitarbeiter des Unternehmens an den Kundenbedürfnissen. \cite{habryn2012}

Durch dabei auftretende Veränderungen und die bereits thematisierte Aufweichung der bisherigen Unternehmensbereiche, müssen auch Mitarbeiter funktionsübergreifend tätig sein.
Damit werden beispielsweise Mitarbeiter, die bisher im Verkauf tätig waren, zu Beratern in Fragen oder Problemen in Zusammenhang mit den Produkten.

\subsection{Operationale Exzellenz im Marketing}
Das Marketing wird durch anfangs geschilderte Bedingungen auf den Märkten zunehmend komplexer und (kosten-)aufwändiger. Aus den Erkenntnissen der Customer Analytics in Verbindung mit den Möglichkeiten der Digitalisierung (z.B. im Sinne direkter Kontaktaufnahme) können Werbemaßnahmen jedoch gezielter ausgeführt werden. Dies ermöglicht den Einsatz personalisierter Werbung bzw. Werbung für bestimmte Kundengruppen (welche bspw. aus einer Segmentierung resultieren). Damit kann zugleich die Effizienz der Werbemaßnahmen gesteigert werden.

\subsection{Kundennähe durch intensiviertes CRM}

In der Studie von Reichheld und Dawkins \cite{reichheld1990}, die eine Vielzahl von Geschäftsbereichen umfasste, konnte gezeigt werden, dass durch eine Erhöhung der Kundenbindung um fünf Prozent der Kunden der gesamte Kundenwert (CLV) um 25 bis 85 Prozent gesteigert werden kann. Techniken der Customer Analytics können also für eine effizientere Gestaltung des CRM genutzt werden. Auf weitere Einzelheiten des CRM in Verbindung mit der Customer Analytics wird in Abschnitt \ref{sec:crm} vertieft eingegangen.
