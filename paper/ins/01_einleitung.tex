\section{Einleitung}
\label{sec:einleitung}

Bereits 1993 konnte in einer Studie von Treacy und Wiersema \cite{wiersema1993} die Relevanz von Customer Analytics gezeigt werden. Ziel dieser dreijährigen und 80 Unternehmen umfassenden Studie war, Merkmale von marktführenden Unternehmen zu identifizieren, die für deren marktbeherrschende Position ausschlaggebend waren. Dabei fand eine Gegenüberstellung von sog. \textquote{value propositions} (den Versprechen der Unternehmen gegenüber den Kunden), mit den jeweiligen internen Strategien, Abläufen und Strukturen statt. Treacy und Wiersema konnten drei dominante Strategien aufzeigen (die sogenannten „value strategies“). Neben den Strategien der operationalen Exzellenz und der Produktführerschaft wurde die Kundennähe als entscheidender Wettbewerbsvorteil identifiziert.

Die operationale Exzellenz wird als eine solche Unternehmensführung bezeichnet, die als „dynamische Fähigkeit zur Rea­lisierung von effektiven und effizienten Kernprozessen der Wertschöpfungskette“ \cite{gleich2008} der Optimierung des Verhältnisses von Qualität und Preis dient. Die Produktführerschaft kann unterdessen als das für den Kunden beste Produkt verstanden werden.

Für eine strategische Ausrichtung im Sinne der Kundennähe muss das Unternehmen jedoch nicht das günstigste oder innovativste Produkt führen, sondern dem Kunden die beste individuell zugeschnittene Lösung bieten.

Die Kundennähe konnte damit als einer von drei entscheidenden Faktoren aufgezeigt werden. Und obwohl ein Fazit der Studie war, dass ein Unternehmen sich in einer der drei Strategien profilieren sollte (während die anderen beiden nur auf einem dem Wettbewerb angemessenen Niveau gehalten werden müssen), war trotzdem eine Veränderung der Unternehmensstrategien von zuvor stark produkt-dominierter Auslegung zu einer kundenzentrierten Sichtweise zu beobachten. Bereits 2010 konnte in einer Studie des IBM Institute for Business Value \cite{ibm2010} die Kundennähe als „number-one priority“ führender Unternehmen identifiziert werden.

Diese Entwicklung wird zusätzlich durch Zeiten umkämpfter und komplexer Märkte verschärft, auf denen Kunden nach zunehmend individuelleren Lösungen verlangen. Die notwendige Veränderung in der Unternehmensstrategie kann nur durch eine Fokussierung auf die Kundenbedürfnisse erreicht werden.

Damit ist ebenfalls eine Veränderung der Marketingmaßnahmen hin zu stärkerer Service- und Beziehungsorientierung zu beobachten, die wiederum zu einer Verschmelzung von zuvor getrennt gehaltenen Unternehmensbereichen führt \cite{olivarogelio2003} und grundlegende Veränderungen der bisherigen Unternehmensstruktur erfordert.

Durch die Erforschung der Kundenperspektive im Sinne der Customer Analytics wird diese Entwicklung überhaupt erst ermöglicht.