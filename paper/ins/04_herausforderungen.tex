\section{Herausforderungen unterschiedlicher Geschäftsmodelle}
\label{sec:herausforderungen}

Je nach Geschäftsmodell können dem Unternehmen verschiedene Kunden zugeordnet werden.

Im Business-to-Business (B2B) Bereich liegen Geschäftsbeziehungen zwischen zwei oder mehr Unternehmen vor, während im Business-to-Consumer (B2C) Umfeld Geschäftsbeziehungen zwischen dem Unternehmen und Privatpersonen vorliegen. \cite{aichele2016}

\subsection{B2B}
Das B2B Umfeld kennzeichnet sich durch kleinere Kundenbestände, bei denen der Überblick über die bestehenden Kunden besser gewahrt werden kann.

Die Beziehungen zwischen den Geschäftspartnern beruhen stärker auf einer persönlichen Ebene, bei der Maßnahmen der direkten Kontaktaufnahme eine bessere Wirkung erzielen. An der Kaufentscheidung eines Produktes sind dennoch mehr Personen beteiligt, da die direkt mit dem Unternehmen kommunizierende Person häufig nicht selbst für eine Entscheidung über einen Kauf befugt ist \cite{aichele2016}. Aus dieser komplexeren Vertriebssituation und komplexeren Produkten entstehen situationsbedingt unregelmäßigere Käufe. Ohne ein konsequentes CRM können die Kundenbeziehungen nicht aufrecht erhalten werden.

Daher hat sich eine Analyse der Beziehungen der Mitarbeiter des Unternehmens und der Angestellten des Kunden bereits als erfolgreich für die Erklärung des Unternehmenserfolgs gezeigt \cite{gummesson2011}. Diese kann durch Methoden der Netzwerkanalyse ausgeführt werden, die auf personenbezogener Ebene stattfinden und somit ein soziales Netzwerk erstellen \cite{habryn2012}. Besonders entscheidend sind dabei Eigenschaften wie die Größe und die Position der Netzwerke \cite{hutt2006}. Aus analytischer Sicht kann das CRM auf Basis der B2B Umgebung durch weitere KPIs, wie dem CLV ergänzt werden.

Zusammenfassend ist also die Customer Analytics im B2B-Segment weniger entscheidend, da das CRM hauptsächlich zur Unterstützung des operativen Geschäfts benötigt wird.

\subsection{B2C}
Im Vergleich zu dem B2B Kontext ist der Kundenbestand der Unternehmen, die hauptsächlich im B2C Bereich agieren, deutlich umfassender. Kunden sind primär Privatpersonen, die sich in weniger komplexen Entscheidungssituationen über weniger komplexe Produkte befinden. Dennoch bringt auch dieses Geschäftsmodell Schwierigkeiten für die Customer Analytics, denn die Entscheidung wird durch eine Vielzahl nicht kontrollierbarer und nicht vorhersagbarer, zumeist unbekannter Faktoren bestimmt. Dies erschwert die Vorhersagbarkeit des Kundenverhaltens. 

Dennoch kommt hier die Customer Analytics stärker zum Tragen, da aufgrund von massiven, sich schnell verändernden Datenbeständen und vieler paralleler Transaktionen keine rein operative Steuerung aus dem persönlichen Kontakt heraus möglich ist.

Für die Einflussnahme auf eine abgrenzte Gruppe oder einzelne Kunden eignen sich beispielsweise Clustermethoden, die eine individuellere Interaktion mit dem Kunden ermöglichen. Damit werden Daten aus CRM Systemen für die Unterstützung analytischer Auswertungen benötigt und müssen fortschreitend durch Methoden der Customer Analytics ausgewertet werden. \cite{aichele2016}
